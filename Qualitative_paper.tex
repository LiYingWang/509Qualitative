% Template for PLoS

\documentclass[10pt]{article}

% amsmath package, useful for mathematical formulas
\usepackage{amsmath}
% amssymb package, useful for mathematical symbols
\usepackage{amssymb}

% hyperref package, useful for hyperlinks
\usepackage{hyperref}

% graphicx package, useful for including eps and pdf graphics
% include graphics with the command \includegraphics
\usepackage{graphicx}

% Sweave(-like)
\usepackage{fancyvrb}
\DefineVerbatimEnvironment{Sinput}{Verbatim}{fontshape=sl}
\DefineVerbatimEnvironment{Soutput}{Verbatim}{}
\DefineVerbatimEnvironment{Scode}{Verbatim}{fontshape=sl}
\newenvironment{Schunk}{}{}
\DefineVerbatimEnvironment{Code}{Verbatim}{}
\DefineVerbatimEnvironment{CodeInput}{Verbatim}{fontshape=sl}
\DefineVerbatimEnvironment{CodeOutput}{Verbatim}{}
\newenvironment{CodeChunk}{}{}

% cite package, to clean up citations in the main text. Do not remove.
\usepackage{cite}

\usepackage{color}

% Use doublespacing - comment out for single spacing
%\usepackage{setspace}
%\doublespacing


% Text layout
\topmargin 0.0cm
\oddsidemargin 0.5cm
\evensidemargin 0.5cm
\textwidth 16cm
\textheight 21cm

% Bold the 'Figure #' in the caption and separate it with a period
% Captions will be left justified
\usepackage[labelfont=bf,labelsep=period,justification=raggedright]{caption}

% Use the PLoS provided bibtex style
\bibliographystyle{plos}

% Remove brackets from numbering in List of References
\makeatletter
\renewcommand{\@biblabel}[1]{\quad#1.}
\makeatother


% Leave date blank
\date{}

\pagestyle{myheadings}
%% ** EDIT HERE **


%% ** EDIT HERE **
%% PLEASE INCLUDE ALL MACROS BELOW

%% END MACROS SECTION


\begin{document}

% Title must be 150 characters or less
\begin{flushleft}
{\Large
\textbf{Examining the explanatory model of the cultural continuity in Taiwan}
}
% Insert Author names, affiliations and corresponding author email.
\\
  Liying Wang\textsuperscript{1*}\\
\bf{1} Department of Anthropology, University of Washington,  Seattle,  Washington,  USA
\\

\textasteriskcentered{} E-mail:   \href{mailto:liying15@uw.edu}{\nolinkurl{liying15@uw.edu}}

\end{flushleft}

\section*{Introduction}\label{introduction}
\addcontentsline{toc}{section}{Introduction}

In this paper, I will focus on a debate about whether the separate
cultural layers in Kiwulan (KWL) site in Northern Taiwan belong to the
same prehistoric ethnic group. The KWL site is located at Ilan city and
near a riverside at the northern margin of the Ilan plain in
Northeastern Taiwan. The site was excavated during 2001 to 2003 by the
Department of Anthropology of National Taiwan University.(Chen 2004)
According to the archaeological remains, the site can be divided into a
lower culture layer and an upper culture layer, dating from 1300B.P. to
800B.P. and 600B.P. to 100B.P.respectively based on radiocarbon dates.
However, there is a debate about whether the archaeological remains from
both layers belong to the same culture or not, because the
archaeological remains show both similar and different patterns. Chen
(2004) argued that these two layers belong to the same culture based on
similar pattern of artifacts. However, Chiu (2004) stated that they
belong to different culture or ethnic group due to the distinct style of
mortuary practice. The difference of these explanations shows distinct
explanatory models in terms of archaeological evidence. The following I
will examine these two explanations, explore the philosophical theories
behinds them, discuss their strength and weakness, and conclude that
combination of these two models might be a better way to explain the
cultural continuity. (Chiu 2004)

\section*{Archaeological evidence}\label{archaeological-evidence}
\addcontentsline{toc}{section}{Archaeological evidence}

The excavation covered an area of 2,800 sqm and revealed hundreds of
thousands of artifacts from a total of 262 archaeological excavation
areas. The following I introduce the archaeological evidence found in
these layers respectively.

\begin{enumerate}
\def\labelenumi{\arabic{enumi}.}
\itemsep1pt\parskip0pt\parsep0pt
\item
  Lower culture layer (1300-800 B.P)
\end{enumerate}

The archaeological evidence include a wide variety of pottery, post
holes, slabs which are associated with households, 35 burials, few iron
knifes and artifacts, grinding stones, slabs, imported ornaments such as
glass beads, agate beads, metal bracelet, glass earrings, fauna remains
such as deer and pigs. There is no much evidence relates to practices of
agriculture, and hunting and gathering are believed as the main
subsistence. For the pottery, there are different kinds of shape and
form, including bowls, vessels, and pots. The most common type of
pottery is the pottery with stamped geometric decoration. Although there
are many types of pottery in the living area, only geometric decoration
pottery and plain vessels found in burials as burial goods. In addition
to pottery, imported ornaments are also the common burial goods. For the
burials, secondary burial is the most common type, and slabs were used
to serve as funerary utensils. The people abandoned this settlement
around 800B.P due to the environmental change (Chen 2007).

\begin{enumerate}
\def\labelenumi{\arabic{enumi}.}
\setcounter{enumi}{1}
\itemsep1pt\parskip0pt\parsep0pt
\item
  Upper culture layer (600-100 B.P.)
\end{enumerate}

The archaeological remains found in this period were more than lower
culture layer. The archaeological evidence includes pottery, imported
ceramics and stonewares, wooden artifacts such as table wares and tools,
grinding stones, metal artifacts such as knives and points, imported
ornaments such as glass beads, agates beads, and metal bracelets, pipes
which made of stone, clay, and metal, local ornaments made of animal
bones, shells, and wood, and fauna assemblage. In addition to artifacts,
90 burials and wooden pole structures were also excavated. These wooden
poles were found aligned in a north-south direction with construction
marks, which were interpreted as the remains of house structures. The
distribution of features shows that the cemetery is located at the north
part of house structures, which indicates the settlement was organized
in some order. According to historical records, this settlement was
believed as the biggest settlement in the 17th century (Nakamura 1938).
The potteries were usually decorated with geometric design, which is
similar to the pottery in earlier period. For the burials, the common
funerary utensils were wooden boards. In addition to imported ornaments,
imported ceramics were also the common burials goods (Chen 2007).

\section*{Links between evidence and
behavior}\label{links-between-evidence-and-behavior}
\addcontentsline{toc}{section}{Links between evidence and behavior}

Chen (2007) focused on the similar archaeological remains in both
layers. In both layer, he pointed out that they find similar burial
goods, such as local pottery, glass beads, and agate beads. Although the
forms and shapes of beads are not completely the same in both layers, he
argued that the distribution and pattern of burial goods is similar.
Agate beads and glass beads usually serve as necklace on deceased, and
some small beads seem they are part of the decoration of clothes. In
addition to beads, pottery is the common burial good in both layers. He
states that the people from these two periods have similar mortuary
practice based on the distribution of burial goods, especially for the
ritual. Moreover, the pattern of pottery in both layers shows the
similar surface treatment, stamped geometric decoration, which shows the
same system of decoration and the same technique for making pots. For
the households, both periods were found post holes and wooden poles,
which indicates they might have similar household structures. He thought
that if we focus on singular artifact, we might think these remains
belong to different groups. But when examining all of the archaeological
remains together, both layers indicate similar pattern of artifacts and
shows some continuity of cultural element.

On the other hand, Chiu (2004) stresses the difference between these two
layers mainly based on mortuary practice. The common burial in lower
cultural layer is secondary burials and the people used slabs as funeral
utensils; however, most burials in upper culture layer are primary with
bodies in flexed position. Moreover, wood is the common funeral utensil
in upper culture layer rather than slabs. Although people from both
layers used beads as burial goods, the forms and shapes shows slightly
different. For example, the lower culture layer was found glass
earrings, which is absent in upper culture layer. Based on Person (1999)
arguments, he stated that mortuary practices are social and political
behaviors. Mortuary practice and associated burial goods are not only
artifacts, but also rules and custom related to the core element of a
culture. Besides, he believed that mortuary practice usually links to
the ritual, the worldview, and the view of life and death of an ethnic
group. The viewpoints for mortuary practice of an ethnic group seldom
change and will pass down from generations to generations unless they
face significant influence. Therefore, he stated that the people from
tow layers belong to different cultures, because their mortuary
practices are different, and there is no relevant evidence shows any
factor which will affect the culture change if they are the same group.

\section{Behavior at different
scales}\label{behavior-at-different-scales}

\section*{Discussion}\label{discussion}
\addcontentsline{toc}{section}{Discussion}

\begin{enumerate}
\def\labelenumi{\arabic{enumi}.}
\itemsep1pt\parskip0pt\parsep0pt
\item
  Context of the monograph's explanatory model
\end{enumerate}

The same culture hypothesis is based on the similar human practice on
daily life, such as making pattery, living, and mortury practive. This
viewpoint stress that we should examine the artifacts as a whole. On the
other hand, different culture hypothesis thinks that mortury parctice
reflects the view of life and death of an ethnic group, which is the
core element in a cultue and seldom changes with time. However, style of
pottery or household tends to be influenced by other culture. This
approach emphasize that mortuary practice can be viewed as counterpart
of society.

\begin{enumerate}
\def\labelenumi{\arabic{enumi}.}
\setcounter{enumi}{1}
\itemsep1pt\parskip0pt\parsep0pt
\item
  Relevant philosophy of science
\end{enumerate}

I think Chen's (2007) argument is based on the inference of best
explanation, because he examined all of the evidence found in these two
periods, and thinks these two periods belongs to the same culture can
fit most evidence we found so far. As Lipton (2000) mentions, a good
explanation should require a cause which can explain both the fact and
other possible phenomenon, and should explain as many phenomena as
possible. I think Chen's argument fulfills these requirements, because
his argument can explain different type of artifacts instead of singular
evidence.

On the other hand, I think Chiu's (2004) argument is based on the D-N
model or unification. D-N model relies the logic connection and
hypothesis building. In this case, it shows that Chiu's argument is
based on the hypothesis that the mortuary practice is the core element
in human culture and seldom changes with time unless they face
significant influence. Also, any ethnic group has distinct mortuary
practice. Because there is no evidence shows any factor that affects the
mortuary practice between these two periods. By applying the D-N model,
he infers that the people from two layers belong to different group
based on the hypothesis of mortuary practice. Besides, I think his
argument is also relative to unification.

\begin{enumerate}
\def\labelenumi{\arabic{enumi}.}
\setcounter{enumi}{2}
\itemsep1pt\parskip0pt\parsep0pt
\item
  Critique I think both arguments need more theoretical framework to
  support the inference. Although Chen's argument can explain more
  evidence in this site, there is no really connection that can explain
  the people belong to the same group. On the other hand, Chiu's
  argument only relies on the mortuary practice, which overlooked other
  possible phenomenon that can provide clues for this question. I think
  his argument should also consider other evidence in order to give
  better explanation.(Chiang 2010)
\end{enumerate}

\section*{Conclusion}\label{conclusion}
\addcontentsline{toc}{section}{Conclusion}

\section*{Reference}\label{reference}
\addcontentsline{toc}{section}{Reference}

Chiu, Hung-Lin 2004 Investigations of Mortuary Behaviors and Cultural
Change of the Kivulan Site in I-Lan County, Taiwan.
宜蘭縣礁溪鄉淇武蘭遺址出土墓葬研究---埋葬行為與文化變遷的觀察,
Department of Anthropology, National Taiwan University.

Chen, Yu-Pei 2007 The Excavation Report of the Ki-Wu-Lan Site 6. I-lan,
Taiwan: Lanyang museum.

2004 The significance of the Kiwulan site for understahnding the
prehistoric period in Ilan Plain.
淇武蘭遺址發掘對蘭陽平原史前研究的意義. Ilan Study.
宜蘭研究第六屆學術研討會, 宜蘭: 宜蘭縣史館 6.

\subsubsection{Comment}\label{comment}

Eerkens, Jelmer W., and Carl P. Lipo. ``Cultural transmission theory and
the archaeological record: providing context to understanding variation
and temporal changes in material culture.'' Journal of Archaeological
Research 15.3 (2007): 239-274.

Tainter Joseph A. ``Mortuary practices and the study of prehistoric
social systems.''Advances in archaeological method and theory (1978):
105-141.

Binford, Lewis R. ``Mortuary practices: their study and their
potential.'' Memoirs of the Society for American Archaeology (1971):
6-29.

Pearson, Michael Parker. ``Mortuary practices, society and ideology: an
ethnoarchaeological study.''Symbolic and structural archaeology 1
(1982): 99-113.

Search other sites, when burial costom changes inferecne of (IBE) Yayoin
Jomon different part of explanation in a same thesis.

Chen, Yu-Pei. 2004. ``The Significance of the Kiwulan Site for
Understahnding the Prehistoric Period in Ilan Plain.'' \emph{Ilan Study}
6.

Chiang, Chihhua. 2010. ``Reconstructing Prehistoric Social Organization:
A Case Study from the Wansan Site, Neolithic Taiwan.'' Dissertation.

Chiu, Hung-Lin. 2004. ``Investigations of Mortuary Behaviors and
Cultural Change of the Kivulan Site in I-Lan County, Taiwan.''
Dissertation, Department of Anthropology, National Taiwan University.

\end{document}

