% Template for PLoS

\documentclass[10pt]{article}

% amsmath package, useful for mathematical formulas
\usepackage{amsmath}
% amssymb package, useful for mathematical symbols
\usepackage{amssymb}

% hyperref package, useful for hyperlinks
\usepackage{hyperref}

% graphicx package, useful for including eps and pdf graphics
% include graphics with the command \includegraphics
\usepackage{graphicx}

% Sweave(-like)
\usepackage{fancyvrb}
\DefineVerbatimEnvironment{Sinput}{Verbatim}{fontshape=sl}
\DefineVerbatimEnvironment{Soutput}{Verbatim}{}
\DefineVerbatimEnvironment{Scode}{Verbatim}{fontshape=sl}
\newenvironment{Schunk}{}{}
\DefineVerbatimEnvironment{Code}{Verbatim}{}
\DefineVerbatimEnvironment{CodeInput}{Verbatim}{fontshape=sl}
\DefineVerbatimEnvironment{CodeOutput}{Verbatim}{}
\newenvironment{CodeChunk}{}{}

% cite package, to clean up citations in the main text. Do not remove.
\usepackage{cite}

\usepackage{color}

% Use doublespacing - comment out for single spacing
%\usepackage{setspace}
%\doublespacing


% Text layout
\topmargin 0.0cm
\oddsidemargin 0.5cm
\evensidemargin 0.5cm
\textwidth 16cm
\textheight 21cm

% Bold the 'Figure #' in the caption and separate it with a period
% Captions will be left justified
\usepackage[labelfont=bf,labelsep=period,justification=raggedright]{caption}

% Use the PLoS provided bibtex style
\bibliographystyle{plos}

% Remove brackets from numbering in List of References
\makeatletter
\renewcommand{\@biblabel}[1]{\quad#1.}
\makeatother


% Leave date blank
\date{}

\pagestyle{myheadings}
%% ** EDIT HERE **


%% ** EDIT HERE **
%% PLEASE INCLUDE ALL MACROS BELOW

%% END MACROS SECTION


\begin{document}

% Title must be 150 characters or less
\begin{flushleft}
{\Large
\textbf{Examining the explanatory model of the cultural continuity of an
archaeological site in Taiwan}
}
% Insert Author names, affiliations and corresponding author email.
\\
  Liying Wang\textsuperscript{1*}\\
\bf{1} Department of Anthropology, University of Washington,  Seattle,  Washington,  USA
\\

\textasteriskcentered{} E-mail:   \href{mailto:liying15@uw.edu}{\nolinkurl{liying15@uw.edu}}

\end{flushleft}

\section*{Introduction}\label{introduction}
\addcontentsline{toc}{section}{Introduction}

In this paper, I will focus on a debate about whether the separate
cultural layers in Kiwulan (KWL) site in Northern Taiwan belong to the
same prehistoric ethnic group. The KWL site is located at Ilan city and
near a riverside at the northern margin of the Ilan plain in
Northeastern Taiwan. The site was excavated during 2001 to 2003 by the
Department of Anthropology of National Taiwan University (Chen 2007).
According to the archaeological remains, the site can be divided into a
lower culture layer and an upper culture layer, dating from 1300B.P. to
800B.P. and 600B.P. to 100B.P.respectively based on radiocarbon dates.
However, there is a debate about whether the archaeological remains from
both layers belong to the same culture or not, because the
archaeological remains show both similar and different patterns. Chen
(2007) argued that these two layers belong to the same culture based on
similar pattern of artifacts. However, Chiu (2004) stated that they
belong to different culture or ethnic group due to the distinct style of
mortuary practice. The difference of these explanations shows distinct
explanatory models in terms of archaeological evidence. The following
sections I will examine these two explanations based on archaeological
evidence, explore the philosophical theories behinds them, discuss their
strength and weakness, and then conclude that Chen's argument might be
more appropriate in this case. However, both explanations still need
clear theoretical framework, and the combination of these two models
might be a better solution to this debate.

\section*{Archaeological evidence}\label{archaeological-evidence}
\addcontentsline{toc}{section}{Archaeological evidence}

The excavation of Kiwulan site covered an area of 2,800 sqm and revealed
hundreds of thousands of artifacts from a total of 262 archaeological
excavation areas. The following I will introduce the archaeological
evidence found in these layers respectively.

\begin{enumerate}
\def\labelenumi{\arabic{enumi}.}
\itemsep1pt\parskip0pt\parsep0pt
\item
  Lower culture layer (1300-800 B.P)
\end{enumerate}

The archaeological evidence include a wide variety of pottery, post
holes, slabs which are associated with households, 35 burials, few iron
knifes and artifacts, grinding stones, slabs, imported ornaments such as
glass beads, agate beads, metal bracelet, glass earrings, fauna remains
such as deer and pigs. There is no much evidence relates to practices of
agriculture. In addition, hunting and gathering are believed as the main
subsistence in this period. For the pottery, there are different kinds
of shape and form, including bowls, vessels, and pots. The most common
type of pottery is the pottery with stamped geometric decoration.
Although there are many types of pottery in the living area, only
geometric decoration pottery and plain vessels found in burials as
burial goods. In addition to pottery, imported ornaments are often used
as burial goods. For the burials, secondary burial is the most common
type, and slabs were used to serve as funerary utensils. The people
abandoned this settlement around 800B.P due to the environmental change
(Chen 2007).

\begin{enumerate}
\def\labelenumi{\arabic{enumi}.}
\setcounter{enumi}{1}
\itemsep1pt\parskip0pt\parsep0pt
\item
  Upper culture layer (600-100 B.P.)
\end{enumerate}

The archaeological remains found in this period were more than lower
culture layer. The archaeological evidence includes pottery, imported
ceramics and stonewares, wooden artifacts such as table wares and tools,
grinding stones, metal artifacts such as knives and points, imported
ornaments such as glass beads, agates beads, and metal bracelets, pipes
which made of stone, clay, and metal, local ornaments made of animal
bones, shells, and wood, and fauna assemblage. In addition to artifacts,
90 burials and wooden pole structures were also excavated. These wooden
poles were found aligned in a north-south direction with construction
marks, which were interpreted as the remains of house structures. The
distribution of features shows that the cemetery is located at the north
part of house structures, which indicates the settlement was organized
in some order. According to historical records, this settlement was
believed as the biggest settlement in the 17th century (Nakamura 1938).
The potteries were usually decorated with geometric design, which is
similar to the pottery in earlier period. For the burials, the common
funerary utensils were wooden boards. In addition to imported ornaments,
imported ceramics were also the common burials goods (Chen 2007).

\section*{Links between evidence and
behavior}\label{links-between-evidence-and-behavior}
\addcontentsline{toc}{section}{Links between evidence and behavior}

Chen (2007) focused on the similar archaeological remains in both
layers. He pointed out the similarity of burial goods, such as local
pottery, glass beads, and agate beads. Although the forms and shapes of
beads are not completely the same in both layers, he argued that the
distribution and pattern of burial goods is similar. Agate beads and
glass beads usually served as necklaces on deceased, and some small
beads reveal that they were part of the decoration of clothes. In
addition to beads, pottery is the common burial good in both layers. He
states that the people from these two periods have similar mortuary
practice based on the distribution of burial goods, especially for
pottery and glass beads. Moreover, the pattern of pottery in both layers
shows the similar surface treatment, stamped geometric decoration, which
also indicates the same system of decoration and technique for making
pots. For the households, both periods were found post holes and wooden
poles, which indicates they might have similar household form and
structure. Chen linked the archaeological evidence with human behaviors
at household level, and viewed these human behaviors as an entity, which
reflects settlement level. He stated that the similarity of artifacts of
two cultural layers indicates the cultural transmission in a singular
culture, in which cultural elements transmits from people to people, and
from generation to generation.

On the other hand, Chiu (2004) stresses the difference between these two
layers mainly based on mortuary practice. The common burial in lower
cultural layer is secondary burials and the people usually used slabs as
funeral utensils; however, most burials in upper culture layer are
primary with bodies in flexed position. Moreover, wood is the common
funeral utensil in upper culture layer rather than slabs. Although
people from both layers used beads as burial goods, the forms and shapes
shows slightly difference. For example, the lower culture layer was
found glass earrings, which is absent in upper culture layer. Based on
Person (1999) arguments, he stated that mortuary practices are social
and political behaviors. Mortuary practices and associated burial goods
are not only artifacts, but also rules and custom related to the core
element of a culture (Pearson 1999). Besides, he believed that mortuary
practice usually links to the ritual, the worldview, and the view of
life and death of an ethnic group. The viewpoints for mortuary practice
of an ethnic group seldom change and will pass down from generation to
generation unless they face significant influence. Therefore, he stated
that the people from two layers belong to different cultures, because
their mortuary practices are different, and there is no relevant
evidence shows any factor that will affect the culture change if they
are the same ethnic group. Regarding the similar burials goods found
from both layers, he argued that the burial goods such as glass beads
were imported materials, which indicates the common interaction between
Northern Taiwan and Southeast China in the prehistoric period rather
than the preference of a culture (Chiu 2004).

\section*{Discussion}\label{discussion}
\addcontentsline{toc}{section}{Discussion}

\begin{enumerate}
\def\labelenumi{\arabic{enumi}.}
\itemsep1pt\parskip0pt\parsep0pt
\item
  Context of the monograph's explanatory model
\end{enumerate}

The same culture hypothesis is based on the similar human practice on
daily life, such as making pattern, living, and mortuary practice. This
viewpoint stresses that we should examine the artifacts as a whole. Chen
stated that if we focus on singular artifact, we might think these
remains belong to different groups. But when examining all of the
archaeological remains together, both layers indicate similar pattern of
artifacts and shows some continuity of cultural element. As a director
of excavation of Kiuwlan site, Chen prefers to interpret the
archaeological remains as a whole, and try to provide an explanation for
these two cultural layers. Also, Chen completed his PhD program at
Graduate school of social and cultural studies of Kyushu University in
Japan around 1999, which tends to interpret the archaeological evidence
based on topology and culture-historical approach. This background
influenced him to focus on the similar cultural elements, and the
relationship of cultural elements and the preference of ethnicity.

On the other hand, different culture hypothesis assumes that mortuary
practice reflects the view of life and death of an ethnic group, which
is the core element in a culture and seldom changes with time. Chiu
argues that although the style of pottery or households show the similar
pattern in both layers, these cultural elements tend to be influenced by
other culture through frequent interaction. This approach emphasizes
that mortuary practice can be viewed as ritual, which reflects the most
important part of a culture. Chiu is interested in mortuary practice and
human remains since he was an graduate student, which leads to such
interpretation based on burials. Although he graduated from the same PhD
program as Dr.~Chen, Chiu was influenced more by explanatory approach to
archaeology. This background leads to the foci of the scientific method
and hypothesis testing. In this case, Chiu used quantitative method to
examine the burials goods, human remains, and forms of mortuary practice
in order to synthesis the discussion of the culture change of Kiwulan
site.

\begin{enumerate}
\def\labelenumi{\arabic{enumi}.}
\setcounter{enumi}{1}
\itemsep1pt\parskip0pt\parsep0pt
\item
  Relevant philosophy of science
\end{enumerate}

I think Chen's (2007) argument is based on the inference of best
explanation, because he examined all of the evidence found in these two
periods, and thought the explanation that two periods belongs to the
same culture can fit most evidence he found so far. As Lipton (2000)
mentioned, a good explanation should require a cause that can explain
both the fact and other possible phenomenon, and should explain as many
phenomena as possible. I think Chen's argument fulfills these
requirements, because his argument can explain different type of
artifacts instead of singular evidence. In addition, Chen focuses on the
culture element of these two layers, which indicates there is a culture
transmission from generation to generation. This assumption to some
extent is based on the culture transmission theory, which stresses the
inheritance of cultural elements to explain human variation (Shennan
2008). The cultural transmission theory can be viewed as unification,
which assumes we can use few core patterns to describe a wide range of
phenomena. In his case, the inheritance of cultural elements is the
common pattern we can found in most societies, and it serves as an
explanation to interpret the similarity of two culture layers.

On the other hand, I think Chiu's (2004) argument is based on the D-N
model and causal mechanism. D-N model relies the logic connection and
hypothesis building. Chiu's explanation shows that his argument is based
on the statement that the ritual is the core cultural element of any
human culture and seldom changes with time unless they face significant
influence. Second statement is that every ethnic group has their
distinct ritual, and the mortuary practice reflects a kind of ritual.
Therefore, by examining the mortuary practice, he stated that it is
possible to understand the culture change of an ethnic group. Because
there is no evidence shows any factor that affects the change in
mortuary practice if these two periods belongs to the same culture. By
applying the D-N model, he infered that the people from two layers
belong to different group. Besides, I think his argument is also
relative to causal mechanism proposed by Glennan, who distinguished
different narrative explanations in terms of explanatory grains (Glennan
2010). In this case, Chiu explained the human remains and mortuary types
in a fine-grained way, and interpret the relationship between mortuary
practices and ethnic group in a coarse-grained way.

\begin{enumerate}
\def\labelenumi{\arabic{enumi}.}
\setcounter{enumi}{2}
\itemsep1pt\parskip0pt\parsep0pt
\item
  Critique
\end{enumerate}

I think both arguments need clear theoretical framework to support their
inferences. Although Chen's argument can explain most archaeological
reamins from the site, there is no explanation for the connection
between the lower culture layer and upper culture layer. For example,
what kind of cultural element can represent an ethnic group, and why
these elements can be transmitted from generation to generation, or from
person to person without change? In other words, it is necessary to
further examine the mechanism of cultural transmission from lower
culture layer to upper culture layer. On the other hand, Chiu's argument
provides the mechanism behinds the mortuary practice and culture change,
which connects the archaeological evidence with the concept of
ethnicity. However, his explanation only relies on the mortuary
practice, which overlooked other possible phenomenon that can provide
clues for the culture continuity. How can we interpret the culture
continuity only based on mortuary practice is the question and it needs
strong theoretical framework and modeling test to support it. Therefore,
I think Chiu's argument should also consider other evidence in order to
give a better explanation. Both arguments show their strengths and
flaws, and I proposed that the combination of these two explanatory
models might provide a better explanation.

\section*{Conclusion}\label{conclusion}
\addcontentsline{toc}{section}{Conclusion}

The culture continuity of Kiwulan site is an important topic in Taiwan,
because if two culture layers belong to the same ethnic group, then the
site can provide a good source to research the long term change of a
culture. Chen's model, which assumes that we can view these two layers
as a same culture, provides an appropriate explanation based on most
archaeological evidence and the similar cultural elements. However, his
explanation lacks of the explanation of mechanism for the culture
transmission. On the other hand, although Chiu's argument only focuses
on the mortuary practice, he examines the evidence both in fine and
coarse-grained way.

I think the combination of these two approaches can provide a better
explanation for the culture continuity. We can examine the relationship
of each type of archaeological evidence and associated human behavior in
a fine-grained way by scientific methods, such as DNA analysis for human
remains, or technical choice in pottery production. In a course-grained
way, we can explore the mechanism of culture change in a ethnic group
based on the general pattern identified by archaeologists from other
places and then formulate a model to test the archaeological evidence
from Kiwulan site. Based on Glennan's causal mechanism approach, we can
combine Chen's culture transmission theory and Chiu's hypothesis testing
in order to obtain a better understanding of the culture continuity of
Kiwulan site.

\section*{Reference}\label{reference}
\addcontentsline{toc}{section}{Reference}

Chen, Yu-Pei. 2007. \emph{The Excavation Report of the Ki-Wu-Lan Site
1}. I-lan, Taiwan: Lanyang museum.

Chiu, Hung-Lin. 2004. ``Investigations of Mortuary Behaviors and
Cultural Change of the Kivulan Site in I-Lan County, Taiwan.''
Dissertation, Taipei: Department of Anthropology, National Taiwan
University.

Glennan, S. 2010. ``Ephemeral Mechanisms and Historical Explanation.''
\emph{Erkenntnis} 72 (2): 251--66.

Nakamura. 1938. ``The Dutch Cencus Record for Indigenous Peoples in
Taiwan.'' \emph{Southern Anthropological Studies} 4 (4): 1--7.

Pearson, Mike P. 1999. \emph{The Archaeology of Death and Burial}. UK:
Sutton: Phoenix Mill.

Shennan, Stephen. 2008. ``Evolution in Archaeology.'' \emph{Annual
Review of Anthropology} 37: 75--91.

\end{document}

